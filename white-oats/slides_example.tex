% slides_example.tex
%
% Example .tex file for the beamer-purdue-oats theme
%
% Copyright (c) 2020 Yang Wang <wang701@purdue.edu>
%
% This file is published under the Creative Commons Attribution 4.0
% International License. For more information, please visit
% https://creativecommons.org/licenses/by/4.0/
%
% See the README.md for usage instructions. This style really only has one
% option: When called with the "oatslogo" option, it will use the OATS logo.

% we want serif math fonts, they look better
\documentclass[pdf, mathserif, aspectratio=169]{beamer}

\usepackage{pgf}  % PGF/TikZ is needed for nice plots

\usepackage[cmex10]{mathtools} % for math equations
\usepackage{amssymb}
\usepackage{bm}

% theorems - can use \begin{IEEEproof} as alternative to \begin{proof}
\usepackage{amsthm}
\renewcommand{\qedsymbol}{$\blacksquare$} % want a black square for proofs

% useful to set this when using Inkscape SVG figures
\graphicspath{{./graphics/}}

% tell beamer to use the purdue-gold theme
\usetheme[sponsorlogo]{purduewhiteoats}

% set title and author
\title{beamer-purdue-oats}
\subtitle{A Beamer template for Purdue OATS}
\author{Yang Wang}

% let's get started
\begin{document}

\begin{frame}[plain]
  \titlepage
\end{frame}

\begin{frame}
  \frametitle{Overview}
  \tableofcontents
\end{frame}

\section{Examples}
\label{sec:examples}

\begin{frame}
  \frametitle{Hello!}
  \framesubtitle{The \texttt{beamer-purdue-oats} template}

  This is the \texttt{beamer-purdue-oats} Theme.

  An itemized list looks as follows:
  \begin{itemize}
  \item Item 1
  \item Item 2
  \end{itemize}

\end{frame}

\begin{frame}
  \frametitle{A Theorem in a Box}

  \begin{theorem}
    The Bessel functions of the first kind $J_{v}(x)$ are the solutions to the
    Bessel differential equation
  \end{theorem}
\end{frame}

\begin{frame}
  \frametitle{A Definition in a Box}

  \begin{varblock}{State-Space Representation}
    A state-space representation is a mathematical model of a physical system as
    a set of input, output and state variables related by first-order
    differential equations or difference equations.
  \end{varblock}

\end{frame}

\begin{frame}
  \frametitle{Figures}

  We can include graphics just like we are used to, for example this block
  diagram of an noise-canceling system:
  \begin{center}
    \input{./graphics/anc_bd.pdf_tex}
  \end{center}
\end{frame}

\section{Plots}
\label{sec:plots}

\begin{frame}
  \frametitle{Plotting is fun!}

  On the following pages, we include two examples on how to include plots:
  \begin{enumerate}
    \item A PDF plot
    \item A PGF/TikZ plot
  \end{enumerate}

  PDF plots are nice, but nothing beats the native look of PGF/TikZ. The source
  code to generate both plots can be found in \texttt{extra/plot\_bessel.py}
\end{frame}

\begin{frame}
  \frametitle{A PDF Plot}
  \begin{center}
    \includegraphics{./plots/bessel.pdf}
  \end{center}
\end{frame}

\begin{frame}
  \frametitle{A PGF/TikZ Plot}
  \begin{center}
    \input{./plots/bessel.pgf}
  \end{center}
\end{frame}

\end{document}
